%!TEX root = ../main.tex

\section{Potentials For Upper Bounds}

Recall the pancake flipping problem. To show the lower bound, we
constructed a potential function $f$ and an input for which
the potential was $n$. We also showed the potential at the
solution is 0. We can also use potentials to prove upper bounds. To do
this we need to show that every configuration has a potential bounded
by $\Delta$, and when we reach the solution we have a configuration
with potential $\delta$. We also need to show that every step of the
algorithm reduces the potential \textbf{at least} $x$. If we can show
this, we have shown that the algorithm has an upper bound of

$$
\frac{\Delta - \delta}{x}
$$

Consider once more the problem of finding the maximum of $n$ elements
in the comparison tree model, and consider the potential function 
$f$ = number of elements that haven't been beaten yet in a comparison.
In other words, $f$ indicates the number of candidates for the
maximum. Initially, $f = n$. When the algorithm concludes,
$f = 1$. Furthermore, any good algorithm that compares only
unbeaten elements will ensure that the potential decreases at least
by 1. Using all of this, we have just shown an upper bound of $n$.


\section{Adversary Method}

One problem that we have with proving lower bounds is that different
algorithms have different worst-cases. Thus we cannot set up one input
and analyze all algorithms on this input and get a good lower bound.
The two techniques we have discussed so far do not create bad inputs
tailored to each algorithm to prove lower bounds. The adversary 
technique does. How can we demonstrate a worst-case input for each of 
a potentially infinite collection of algorithms?

The trick is to design a strategy for the adversary that fields
questions made by any algorithm and answers these questions in such a
way that the progress of the algorithm to the solution is slowed down
as much as possible. A strategy for an adversary defines an answer for
each possible question in each possible "state", where a state is the set of questions that have been asked and answers that have been given to these questions. Thus it can be seen that the adversary is adaptively creating a worst-case input for the algorithm. One constraint is that the adversary should give a consistent set of answers, i.e., it should be possible to create an input for which all of the adversary’s answers are correct.

\subsection{Edge Probe Model}

Consider a graph $G = (V, E)$, and suppose we ask a question about the
graph. In the edge probe model our
primitive operation is to probe a pair of vertices $(i, j)$ to learn
if there is an edge connecting $i$ and $j$ or not. Each probe costs 1
step. Using the edge probe model, the adversary will answer the
algorithm's probes, making sure to slow down the algorithm as much as
possible. We can deduce that an immediate upper bound is 
$\binom{n}{2}$ on the number of probes required (probing every vertex
against every vertex). 

\subsubsection{Is The Graph Connected?}

In this problem, the adversary maintains the connected components of
the graph created so far. For a query $(i, j)$, the adversary answers
yes if and only if the connection between $i$ and $j$ is the last
possible connection between the component of $i$ and the component of
$j$.

\subsection{Evasiveness}

\begin{definition}
    A graph property that requires any algorithm to query all pairs in
    the edge probe model is called evasive.
\end{definition}

\begin{definition}
    (Informal) A property of a graph is monotone if the addition of
    more edges makes the property more true.
\end{definition}

This leads to the following conjecture.

\begin{conjecture}
    Every monotone graph property is evasive.
\end{conjecture}

We will now show informally that knowing if the graph is connected is
an evasive property. We will need the following lemma.

\begin{lemma}\label{lemma:lem1}
    Fix the above adversary strategy. For any algorithm $A$, at any
stage, let $C$ be a connected component that has been revealed by the
probes that $A$ has made. Then $A$ must have probed every pair of
distinct vertices in $C$.
\end{lemma}

\begin{theo}[]{theo:theo1}
Under the adversary strategy, any correct
algorithm must perform $\binom{n}{2}$ queries to determine if
a graph is connected. 
\end{theo}

\begin{prf}[]{proof:theo1}
(Informal) By induction. Suppose by induction we know $C_1$ and $C_2$
are connected components by probing every pair of vertices in $C_1$
and $C_2$ (using Lemma~\ref{lemma:lem1}). When we connect $C_1$ and
$C_2$ to form $C = C_1 \cup C_2$, since we are using the adversary
method, all vertices between $C_1$ and $C_2$ have been probed,
maintaining the inductive invariant.
\end{prf}

\section{Divide and Conquer}

This is a technique to solve problems. It consists of three steps.

\begin{itemize}
    \item Divide: divides the problem into roughly equal-sized pieces.
    \item Conquer: solves each piece.
    \item Combine: combines the solutions to the pieces.
\end{itemize}

The classic problem to solve using divide and conquer is mergesort.

\subsection{Mergesort}

\begin{itemize}
    \item Divide: trivial, split the array into two subarrays.
    \item Conquer: recurse.
    \item Combine: this is the merge algorithm. Given sorted arrays
    $A$ and $B$, this algorithm merges $A$ and $B$ to give a sorted
    array $C$ (see 
    \href{https://en.wikipedia.org/wiki/Merge_algorithm}{here} for
    more info). The number of comparisons is $2n - 1$. What is the
    lower bound for this algorithm? Using the adversary strategy, the
    adversary will initially assume that $a_1 < b_1 < a_2 < b_2 ...$.
    The algorithm will proceed to compare adjacent elements,
    satisfying the adversary's plan. If at any point the algorithm
    tries to shortcut by comparing non-adjacent elements, the
    adversary can swap to slow down the algorithm. Therefore, the
    lower bound for this algorithm is again the number of adjacent
    comparisons which is $2n - 1$.
\end{itemize}


Let $T(n)$ be the number of comparisons performed by mergesort on
an array of $n$ elements. We know $T(1) = 0$. Then for a general $n$
we have that

$$T(n) \leq 2T(\frac{n}{2}) + 2n$$




