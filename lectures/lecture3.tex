%!TEX root = ../main.tex

\section{Master Theorem}

We now introduce the master theorem to solve recurrence relations.


\begin{theo}[Master Theorem]{theo:theo2}
Suppose $T(n) \leq aT(\frac{n}{b}) + n^c$, where $a$ is the number of
subproblems, each of size $n/b$.
\begin{itemize}
    \item If $\frac{a}{b^c} < 1$, then $T(n) = O(n^c)$
    \item If $\frac{a}{b^c} = 1$, then $T(n) = O(\log_b(n) \cdot n^c)$
    \item If $\frac{a}{b^c} > 1$, then $T(n) = O(n^{\log_b(a)})$
\end{itemize}
\end{theo}

We can picture the Master Theorem using a recursion tree as in 
Figure~\ref{fig:recursion-tree}. The key intuition is that the number
of leaves is $n^{\log_b(a)}$.

\begin{figure}
    \centering
    \includegraphics[width=0.5\textwidth]{figures/recursion-tree.png}
    \caption{Recursion tree to visualize the master theorem}
    \label{fig:recursion-tree}
\end{figure}

Going back to mergesort, we have that

$$
T(n) \leq 2T(\frac{n}{2}) + kn
$$

In this case we have that $a = 2$, $b = 2$, and $c = 1$. Using
the Master Theorem (Theorem~\ref{theo:theo2}), we learn
$T(n) = O(n\log(n))$.

\section{Quicksort}

Another classic divide and conquer algorithm is quicksort. We have
as input an array of $n$ integers, and as output we have the integers
in sorted order. The idea is to pick an element in the array to act as
the pivot element $x$. By comparing each other element with the pivot,
we
create a partition with $S = \{ y : y < x \}$ and $L = \{y : y > x \}$
and recursively quicksort $S$ and $L$.

A minor problem with quicksort is that it has worst-case running time
of $O(n^2)$ (consider when the input is already sorted). That is why
we sometimes consider the average-case complexity. We won't consider
average case complexity in this course, as we are pessimistic people.

There is a component of quicksort that is random (choosing the pivot).
Therefore in general we need to talk about expected worst case
behavior. In the algorithm, we pick the pivot uniformly at random from
the elements to be sorted.

Let $X$ be the total number of comparisons performed by quicksort.
Since we have randomized the algorithm, $X$ is a random variable.
Let $X_{ij}$ be the random variable that is 1 if the $i$-th smallest
element and $j$-th smallest element are compared and 0 otherwise.
How do we relate $X$ to $X_{ij}$? Well, we have that

$$
X = \sum_{i<j} X_{ij}
$$

Since we are studying the expected worst case behavior, we need to
analyze the expectation of $X$. By linearity of expectation, we have
that 

\begin{align}
E[X] &= E[\sum_{i<j} X_{ij}] \\
     &= \sum_{i<j} E[X_{ij}] \\ 
     &= \sum_{i<j} P(X_{ij} = 1) \\
     &= \sum_{i<j} \frac{2}{j - i + 1} \\
     &= 2 \sum_{d = 1}^{n - 1} \frac{n-d}{d+1}, \;\; \text{where $d =
     j -
     i$} \\
     &= 2n\sum_{d = 1}^{n - 1}\frac{1}{d+1} - 2\sum_{d = 1}^{n -
     1}\frac{d}{d+1} \\
     &\leq 2n\sum_{d = 1}^{n - 1}\frac{1}{d+1} \\
     &\implies O(n\log(n))
\end{align}

where (2.4) follows because there are $j - i + 1$ possibilities that
will determine the
values of $X_{ij}$, and only 2 compare $i$ with $j$ and (2.8) follows
because (2.7) follows a harmonic series of order $\log(n)$.

\section{Order Statistics}

In order statistics we look at finding the $r$-th smallest element in
an array. We immediately see a solution that takes $O(n\log(n))$ by
first sorting the array and then choosing the $r$-th index. But can we
do better?

\begin{definition}
    An element $x$ in an array $A$ has rank $k$ if $x$ is the $k$-th
    smallest element in $A$.
\end{definition}

\subsection{Quick Select}

As with quicksort, choose a random pivot and split the remaining
elements into $S$ and $L$. If $|S| \geq r$, then we simply recurse on
$S$. If $|S| = r - 1$, then the pivot is the output. Finally, if
$|S| = k < r - 1$, then we recurse in $L$ but now we want to find
the $k + 1$-th smallest element in $L$.

We will not do the expected worst case analysis, but one can show that
it is $O(n)$.

\subsection{Deterministic Linear Time Selection}

Consider $n$ elements.
 
\begin{enumerate}
    \item Create $n/5$ groups of 5 elements each.
    \item Find the median of each group (this can be done in 6
    comparisons). We now have performed $6n/5$ comparisons or equivalently $1.2n$.
    \item Recursively find the median $m*$ of the medians, which takes
    $T(5/n)$. We claim $m*$ has rank somewhere in the middle. More
    precisely, $3n/10$ are smaller than $m*$ and $3n/10$ are bigger,
    i.e., $0.3n \leq \text{rank}(m*) \leq 0.7n$.
    \item Pivot using $m*$
    \item Recurse with no more than $T(7n/10)$. The recurrence
    relation for this algorithm is 
    $$
    T(n) \leq T(\frac{n}{5}) +T(\frac{7n}{10}) + 2.2n
    $$
    which has solution $T(n) \leq n$ for $n \geq 22$.
\end{enumerate}


\begin{remark}
    Honestly, I recommend looking at
\href{http://www.cs.ust.hk/mjg_lib/Classes/COMP3711H_Fall16/lectures/Deterministic_Selection_Notes.pdf}{this} resource for a better explanation.
\end{remark}

