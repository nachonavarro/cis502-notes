%!TEX root = ../main.tex

\section{More Convex Hull}

Convex hulls have many domains. One such examples is in collision detection.

\subsection{Quickhull}

(Spent 5 minutes in class, not very important).
The idea is to get the
point with the minimum $x$-coordinate and the
point
with the maximum $x$-coordinate. Draw a line between these two points,
and hopefully we have divided the points into two sets. Repeat this
for subproblems and merge.

\section{Graphs}

Let $G = (V, E)$ where $V$ is a finite set of vertices and $E$ is a
set of unordered pairs of vertices. There are two representations of
graphs: adjacency lists and adjacency matrices. 

\begin{definition}
    An adjacency list is an array of size $|V|$ where entry $i$ is a
    linked list consisting of the neighbors of vertex $i$.
\end{definition}

\begin{definition}
    An adjacency matrix is a $|V| \times |V|$ matrix $M$ where
    $M_{ij} = 1$ if $(i, j) \in E$ and 0 otherwise. Note the adjacency
    matrix is symmetric for undirected graphs.
\end{definition}

The default representation of graphs is adjacency lists. Note we will
not need to say every time that $G$ is undirected, this
is
implied.

\begin{definition}
    An edge $(u, v)$ has endpoints $u$ and $v$ or equivalently, is
    incident on
    vertices $u$ and $v$. We also say $u$ and $v$ are adjacent if $
    (u, v) \in E$.
\end{definition}

\begin{definition}
    The degree of a vertex $v$ is the number of edges incident on $v$,
    denoted as $d(v) \in \mathbb{N}$.
\end{definition}

\begin{definition}
    A path in a graph $G$ is a sequence of vertices $v_1, v_2, ...,
    v_k$ such that $(v_i, v_{i+1}) \in E$ for all $i = 1 \to k - 1$.
\end{definition}

\begin{definition}
    A path is simple if it does not repeat vertices.
\end{definition}

\begin{definition}
    A (simple) cycle is a sequence of vertices $v_1, v_2, ...,
    v_k, v_0$ such that $(v_i, v_{i+1}) \in E$ for all $i = 1 \to k -
    1$ and $(v_k, v_1) \in E$ with the $v_i$ distinct except $v_1$.
\end{definition}

\begin{definition}
    A graph is acyclic if there are no cycles.
\end{definition}

\begin{definition}
    We say $u$ and $v$ are connected if there is a path between them.
    $G$ is connected if for all $u, v \in V$ there is a path
    connecting $u$ and $v$.
\end{definition}

\begin{definition}
    The connected components of $G$ are the maximal subsets of $G$
    that are pairwise connected.
\end{definition}

\begin{remark}
    Connectedness is an equivalence relation.
\end{remark}

\subsection{Trees}

\begin{definition}
    A tree is a connected, acyclic graphs.
\end{definition}

\begin{definition}
    (Inductive definition) A rooted tree is either
    \begin{itemize}
        \item A graph consisting of a single vertex $v$ with $v$ as
        the root.
        \item If $(T_1, r_1), (T_2, r_2), ..., (T_k, r_k)$ are rooted
        trees, then the tr ee $(T, r)$ consisting of a new node $r$ as
        the root and edges $(r, r_1), ..., (r, r_k)$ is a rooted tree.
    \end{itemize}
\end{definition}

Typically problems in this domain can be solved with structural
induction. An example of structural induction is the following:

\begin{lemma}
    Any tree of $n$ nodes has $n - 1$ edges.
\end{lemma}

\begin{proof}
    We prove that any rooted tree on $n$ nodes has $n - 1$ edges which
    implies our lemma. To prove this we first prove the base case.
    Indeed, the statement holds for any single node rooted tree with
    no edges. For the inductive step, we want to prove this for a
    rooted tree built up from $(T_1, r_1), (T_2, r_2), ..., (T_k,
    r_k)$. Assume the statement is true for all the trees $T_i$.
    Tree $T_i$ has $n_i$ nodes. So then $T$ has $1 + \sum_i^k n_i$
    nodes. By the inductive assumption, there are $n_i - 1$ edges in
    $T_i$, so the total number of edges in $T$ is 
    $k + \sum_i^k (n_i - 1) = \sum_i^k n_i$ which is one less than the
    number of nodes, proving our lemma.
\end{proof}

\subsection{Traversals}

How can we traverse a rooted tree? In other words, how can we visit
all the nodes of a tree? We introd  uce post order traversal.

\begin{algorithm}
\caption{Post order traversal}
\begin{algorithmic}
\REQUIRE{$T$ is a tree with root $r$}
\FOR{child $c$ of $r$}
\STATE{Post order traversal of sub tree with root $c$}
\ENDFOR
\STATE{Visit $r$}
\end{algorithmic}
\end{algorithm}
















