%!TEX root = ../main.tex

Note: Some of Lecture 10 is at the end of Lecture 9's notes for
continuation.

\section{Matroids}

\begin{definition}
    A matroid is a pair $(S, I)$ where $S$ is a finite set and 
    $I$ is some set of subsets of $S$, designated
        "independent sets", such that
    \begin{itemize}
        \item $I$ satisfies the hereditary property. More formally, 
        if $A \subset B$ and $B \in I$, then $A \in I$.
        \item $I$ satisfies the exchange property. More formally,
        if $A, B \in I$ and $|A| < |B|$, then there exists some
        $x \in B - A$ such that $A \cup \{x\} \in I$.
    \end{itemize}
\end{definition}

All maximal independent sets in a matroid have the same size. A greedy
algorithm is optimal for finding a maximum weight maximum independent
set in a matroid.

\subsection{Graphic Matroid}

Let $M = (S, I)$ be a matroid. Given a weighted, connected, undirected
graph $G = (V, E)$ where $S = E$
\begin{itemize}
    \item $A \subset S$ is called independent if and only if $A$ is
    acyclic.
    \item If $B \in I$ and $A \subset B$, then $A \in I$, and $M$ is
    hereditary.
\end{itemize}

\begin{lemma}
    If $A, B \in I$ and $|A| < |B|$, then there exist $e \in B - A$
    such that $A \cup \{e\} \in I$.
\end{lemma}

\begin{proof}
    The graph on $B$ has $n - |B|$ components, and the graph on
    $A$ has $n - |A|$ components. Therefore, $n - |A| > n - |B|$.
    $B$ must contain at least one edge $(u, v)$, where $u$ and
    $v$ lie in different connected components of the graph formed by
    $A$. This edge $(u, v)$ can be added to $A$ to get a larger
    independent set.
\end{proof}

\subsection{Maximum Weight Minimum Spanning Tree}

Our input is a connected, weighted, undirected graph $G = (V, E)$
with some weight function $w : E \to \mathbb{R}$. Our goal is to 
find a maximum weight spanning tree in $G$.

Because this is the problem of finding a maximum weight maximum
independent set in a graphic matroid, the greedy algorithm (Kruskal's)
is optimal.

\begin{definition}
    A cut in a graph is defined by a non-trivial partition
    $V_1 \cup V_2$ of $V$. The cut $(V_1, V_2)$ is the set of edges
    that have exactly one endpoint in $V_1$.
\end{definition}












